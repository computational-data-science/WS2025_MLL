\documentclass[12pt,a4paper]{article}
\usepackage[utf8]{inputenc}
\usepackage[ngerman]{babel}
\usepackage{graphicx}
\usepackage{booktabs}
\usepackage{caption}
\usepackage{float}
\usepackage{amsmath}
\usepackage{geometry}
\geometry{margin=2.5cm}

\title{Wer überlebt den Untergang? \\ Erste Analyse des Titanic-Datensatzes}
\author{Max Mustermann}
\date{\today}

\begin{document}

\maketitle

\section*{1. Einleitung oder Datensatzbeschreibung}

Der Titanic-Datensatz enthält Informationen zu Passagieren des berühmten Schiffsunglücks von 1912. Ziel dieser Analyse ist es, erste Vorhersagen zur Überlebenswahrscheinlichkeit eines Passagiers zu treffen – basierend auf Merkmalen wie Alter, Geschlecht oder Ticketklasse. 

Die Zielvariable \texttt{survived} ist binär kodiert: \texttt{0} für nicht überlebt, \texttt{1} für überlebt.

\section*{2. Erste explorative Analyse}

Eine erste Analyse zeigt deutliche Unterschiede in den Überlebenschancen zwischen Männern und Frauen.

\begin{figure}[H]
    \centering
    \includegraphics[width=0.65\textwidth]{images/survival_by_sex.png}
    \caption{Überlebensrate nach Geschlecht}
\end{figure}

Auch die Ticketklasse (\texttt{pclass}) scheint mit dem Überleben zusammenzuhängen: Je höher die Klasse (1. Klasse), desto höher die Überlebenswahrscheinlichkeit.

\section*{3. Erstes Modell: Logistische Regression}

Für ein erstes Klassifikationsmodell wurde eine logistische Regression trainiert, basierend auf den Merkmalen \texttt{age}, \texttt{fare} und \texttt{sex}. Fehlende Alterswerte wurden durch den Median ersetzt.

Das Modell wurde auf einem Testdatensatz evaluiert. Die folgende Konfusionsmatrix zeigt die Vorhersagequalität:

\begin{figure}[H]
    \centering
    \includegraphics[width=0.6\textwidth]{images/confusion_matrix.png}
    \caption{Konfusionsmatrix: Logistische Regression}
\end{figure}

Das Modell erreicht eine Genauigkeit (Accuracy) von rund 74{,}8\,\%. Damit kann es den Überlebensstatus der Passagiere bereits in vielen Fällen korrekt vorhersagen, zeigt aber noch deutliche Verbesserungspotenziale. 
Auffällig ist, dass das Modell Nicht‑Überlebende häufiger richtig klassifiziert als Überlebende.

\section*{4. Fazit und nächste Schritte}

Die ersten Ergebnisse sind vielversprechend: Schon ein einfaches Modell mit wenigen Features liefert plausible Vorhersagen. Als nächster Schritt sollen:

\begin{itemize}
    \item weitere Merkmale berücksichtigt werden (\texttt{embarked}, \texttt{family size}),
    \item Entscheidungsbäume mit Feature-Importance ausprobiert werden,
    \item und ROC/AUC als zusätzliche Metrik eingeführt werden.
\end{itemize}

Langfristig könnte ein kombiniertes Modell entstehen, das auf mehreren Klassifikatoren basiert (z.\,B. Random Forest).

\end{document}

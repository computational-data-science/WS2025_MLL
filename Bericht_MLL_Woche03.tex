\documentclass[12pt,a4paper]{article}

% ----- Sprache, Encoding, Typografie -----
\usepackage[ngerman]{babel}
\usepackage[T1]{fontenc}
\usepackage[utf8]{inputenc}
\usepackage{lmodern}
\linespread{1.06}
\usepackage[a4paper,margin=2.5cm]{geometry}

% ----- Mathe, Tabellen, Grafiken -----
\usepackage{amsmath, amssymb}
\usepackage{booktabs}
\usepackage{graphicx}
\graphicspath{{figs/}}
\usepackage{caption}
\usepackage{subcaption}

% ----- Literatur (natbib) -----
\usepackage[round,authoryear]{natbib}
%\usepackage[square,numbers]{natbib}
\bibliographystyle{apalike}

% ----- Querverweise -----
\usepackage[hidelinks]{hyperref}
\usepackage[nameinlink,capitalise]{cleveref}

% ----- Sonstiges -----
\newcommand{\todo}[1]{\textit{\textbf{[TODO: #1]}}}

% =========================================================
\begin{document}

\begin{titlepage}
  \centering
  {\Large Technische Hochschule Ingolstadt}\\[6pt]
  {\large Studiengang: Data Science in Technik und Wirtschaft}\\[28pt]

  {\huge \textbf{Wer überlebt den Untergang?}}\\[10pt]
  {\LARGE \textbf{Erste Analyse des Titanic-Datensatzes}}\\[32pt]

  {\large Seminararbeit Machine Learning Lab}\\[24pt]

  \begin{tabular}{ll}
    Autor: & Max Mustermann \\
    Matrikelnummer: & 12345678 \\
    Abgabedatum: & 21.10.2025\\[32pt]
  \end{tabular}

  % ----- Abstract -----
  \begin{minipage}{0.85\textwidth}
    \textbf{Abstract} \\[4pt]
    Die vorliegende Arbeit analysiert den historischen Titanic-Datensatz, um mithilfe von Klassifikationsverfahren vorherzusagen, welche Passagiere das Unglück überlebten. Auf Basis soziodemografischer und ökonomischer Merkmale wurden drei Modelle – logistische Regression, Entscheidungsbaum und Random Forest – verglichen. Der Random Forest erzielte mit einer Genauigkeit von rund 79 \% die beste Vorhersageleistung, während die logistische Regression durch ihre einfache Interpretierbarkeit überzeugte. Die Ergebnisse zeigen, dass Ensemble-Verfahren komplexe Muster besser erfassen, jedoch an Transparenz verlieren. Damit verdeutlicht die Analyse exemplarisch, wie Data-Science-Methoden genutzt werden können, um reale Entscheidungsprozesse datenbasiert zu verstehen und zu bewerten.
  \end{minipage}

\end{titlepage}

% =========================================================
\section{Einleitung}

Am 15. April 1912 sank die RMS Titanic auf ihrer Jungfernfahrt nach einer Kollision mit einem Eisberg im Nordatlantik.
Mehr als 1500 der rund 2200 Menschen an Bord kamen dabei ums Leben, womit das Unglück zu den größten zivilem Schiffs­katastrophen des 20. Jahrhunderts zählt \citep{BBC2024Titanic, BritannicaTitanic}. Der Untergang der Titanic steht bis heute sinnbildlich für das Spannungsfeld zwischen technologischem Fortschritt, sozialer Ungleichheit und menschlicher Fehlbarkeit.

Der erhaltene Passagierdatensatz bietet eine seltene Möglichkeit, historische Ereignisse mit modernen Methoden der Datenanalyse zu untersuchen.
Die in vielen Data-Science-Kursen genutzte Version des Datensatzes basiert auf öffentlich zugänglichen Listen der Passagiere und Crew­mitglieder und wird von der Plattform Kaggle bereitgestellt \citep{KaggleTitanic}. Er enthält Merkmale wie Alter, Geschlecht, Passagierklasse und Ticketpreis und erlaubt damit die Untersuchung soziodemografischer und ökonomischer Einflussfaktoren auf das Überleben.

Diese Arbeit untersucht, welche dieser Merkmale die Überlebenswahrscheinlichkeit am besten vorhersagen. Dazu werden drei Klassifikationsverfahren – logistische Regression, Entscheidungsbaum und Random Forest – verglichen. Neben der Vorhersagegenauigkeit steht auch die Interpretierbarkeit der Modelle im Mittelpunkt, um zu verstehen,
wie Data-Science-Methoden genutzt werden können, um reale gesellschaftliche Fragestellungen datenbasiert zu beleuchten.

% =========================================================
\section{Datenanalyse und Vorbereitung}

% =========================================================
\bibliography{refs}

\end{document}